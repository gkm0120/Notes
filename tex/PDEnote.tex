%!TEX program = xelatex
\documentclass[cn,hazy,blue,14pt,screen]{elegantnote}
\title{Partial Differential Equations}

\author{\href{http://github.com/gkm0120}{\LARGE 戈孔明}}

%\institute{湖南师范大学}

\version{2.0}

\date{\zhtoday}

\usepackage{array}

\begin{document}

\maketitle

\centerline{
  \includegraphics[width=0.2\textwidth]{logo.jpg}
}

\leftline{\quad \quad 由于学识有限,本文缺点和错误在所难免,欢迎批评指正,如有雷同,纯属巧合。}

\leftline{联系方式:\href{mailto:gkm0120@163.com}{gkm0120@163.com}}

\newpage

	\section{偏微分方程解答}
	\begin{example}
		介绍双曲抛物椭圆方程的形式与特点.
		\begin{proof}[Solution]
%			情形1. \\
%			1. 弦振动(双曲)
%				\[\begin{aligned}
%				&\text{一维}: u_{tt}-a^2u_{xx}=f(x,t)\\
%				&\text{二维}: u_{tt}-a^2(u_{xx}+u_{yy})=f(x,y,t)\\
%				&\text{三维}: u_{tt}-a^2(u_{xx}+u_{yy}+u_{zz})=f(x,y,z,t)
%				\end{aligned}
%				\label{1.1} \tag{1.1}
%				\]
%			2. 热传导(抛物)
%				\[\begin{aligned}
%				&\text{一维}: u_{t}-a^2u_{xx}=f(x,t)\\
%				&\text{二维}: u_{t}-a^2(u_{xx}+u_{yy})=f(x,y,t)\\
%				&\text{三维}: u_{t}-a^2(u_{xx}+u_{yy}+u_{zz})=f(x,y,z,t)
%				\end{aligned}
%				\label{1.2} \tag{1.2}
%				\]
%			3.1 Laplace方程(椭圆)
%				\[\begin{aligned}
%				&\text{一维}: a^2u_{xx}=0\\
%				&\text{二维}: a^2(u_{xx}+u_{yy})=0\\
%				&\text{三维}: a^2(u_{xx}+u_{yy}+u_{zz})=0
%				\end{aligned}
%				\label{1.3} \tag{1.3}
%				\]				
%			3.2 Possion方程(椭圆)
%				\[\begin{aligned}
%				&\text{一维}: a^2u_{xx}=f(x)\\
%				&\text{二维}: a^2(u_{xx}+u_{yy})=f(x,y)\\
%				&\text{三维}: a^2(u_{xx}+u_{yy}+u_{zz})=f(x,y,z)
%				\end{aligned}
%				\label{1.4} \tag{1.4}
%				\]
%			情形2. \\
			1. 一般的含有两个自变量的线性偏微分方程可写成如下形式:
			\[au_{xx}+2bu_{xy}+cu_{yy}+du_x +eu_y +gu=f,\label{1.1} \tag{1.1}\]			
			其中$a,b,c,d,e,g$和$f$都是$x,y$的已知函数,且在$xOy$平面上具有二阶连续偏导数.作可逆自变量变换
			\[\begin{array}{l}
			\left\{\begin{aligned}
			&\xi=\varphi(x,y),\\
			&\eta=\psi(x,y), 
			\end{aligned}\right.
			\end{array}
			\label{1.2} \tag{1.2}
			\] 
			记Jacobi行列式
			\[J=\left|
			\begin{array}{cc}
				\varphi(x) & \varphi(y) \\ 
				\psi(x) & \psi(y)
			\end{array} \right| \ne 0,
			\]
			则在可逆自变量变换\eqref{1.2},方程\eqref{1.1}变为如下形式:
			\[
			Au_{xx}+Bu_{xy}+Cu_{yy}+Du_x+Eu_y+Gu=F,
			\label{1.3} \tag{1.3}
			\]
			其中
			\[
			A(\xi ,\eta)=a^2\varphi^2_x+2b\varphi_x\varphi_y+c\varphi^2_y,
			\label{1.4} \tag{1.4}
			\]
			\[
			B(\xi ,\eta)=a\varphi_x \psi_x +b (\varphi_x\psi_y+\varphi_y \psi_x) +c\varphi_y \psi_y,
			\label{1.5} \tag{1.5}
			\]
			\[
			C(\xi ,\eta)=a^2\psi^2_x+2b\psi_x\psi_y+c\psi^2_y,
			\label{1.6} \tag{1.6}
			\]			
			设$\Delta=b^2-ac$为\eqref{1.1}的判别式,$\Delta^{\prime}=B^2-AC$为\eqref{1.3}的判别式,计算知
			\[
			\Delta^{\prime}=J^2\Delta,
			\label{1.7} \tag{1.7}
			\]			
			即在可逆自变量变换下,方程判别式符号保持不变.
			
			当$\Omega \in {\rm R}^2$是一区域,$(x_0,y_0)\in \Omega$时,\\
			(1). 若$\Delta(x_0,y_0)>0$,则称方程\eqref{1.1}在点$(x_0,y_0)$处为\textbf{双曲型偏微分方程};\\
			(2). 若$\Delta(x_0,y_0)=0$,则称方程\eqref{1.1}在点$(x_0,y_0)$处为\textbf{抛物偏微分方程};\\
			(3). 若$\Delta(x_0,y_0)<0$,则称方程\eqref{1.1}在点$(x_0,y_0)$处为\textbf{椭圆型偏微分方程};\\
			若在$\Omega$的每一点处,方程满足对应的$\Delta$判别式,则称\eqref{1.1}在$\Omega$内为对应的偏微分方程.
			
			2. 考虑主部具有常系数的多个自变量的二阶线性偏微分方程
			\[
			\sum_{i,j=1}^{n}a_{ij}u_{x_i x_j}+\sum_{i=1}^{n}u_{x_i}+c(x_1,\cdots,x_n)u=f(x_1,\cdots,x_n)
			\label{2.8} \tag{2.8}
			\]
			其中$a_{ij}=a_{ji}$为常数.
			作非奇异线性变换
			\[\left(
			\begin{array}{c}
				\alpha_1 \\ 
				\alpha_2 \\ 
				\vdots \\ 
				\alpha_n
			\end{array}\right) 
			= \textbf{B} \left(
			\begin{array}{c}
			\beta_1 \\ 
			\beta_2 \\ 
			\vdots \\ 
			\beta_n
			\end{array}\right)
			\]
			其中\textbf{B}是一个可逆矩阵,使得$\textbf{B}^\top 444\textbf{AB}=diag(\lambda_1,\cdots,\lambda_n)$.
			则\eqref{2.8}可化为
			\[
			\sum_{i=1}^{n}\lambda_{i}u_{y_i y_i}+\sum_{i=1}^{n}B_i(y_1,\cdots,y_n)u_{y_i}+C(y_1,\cdots,y_n)u=F(y_1,\cdots,y_n)
			\label{2.9} \tag{2.9}
			\]
			称\eqref{2.9}为\eqref{2.8}的\textbf{标准型}.
			
			若\eqref{2.9}中的$n$个系数$\lambda_i(i=1,\cdots,n)$全是1或-1,则称\eqref{2.8}为\textbf{椭圆型偏微分方程};若\eqref{2.9}中的$n$个系数$\lambda_i(i=1,\cdots,n)$中有一个为1,$n-1$个为-1或一个为-1, $n-1$个为1,则称\eqref{2.8}为\textbf{双曲型偏微分方程};若\eqref{2.9}中的$n$个系数$\lambda_i(i=1,\cdots,n)$全不为零,但取1或-1的个数超过1,则称\eqref{2.8}为\textbf{超双曲型偏微分方程};若\eqref{2.9}中的$n$个系数$\lambda_i(i=1,\cdots,n)$中有一个为零,其余全是1或-1,则称\eqref{2.8}为\textbf{抛物型偏微分方程};

		\end{proof}
	\end{example}

	\begin{example}
		用分离变量法求初边值问题.
		\[\begin{array}{l}
		\left\{\begin{aligned}
		&u_{tt}-a^2u_{xx}=0, \quad 0< x < l,\quad t>0,\\
		&\left.u\right|_{t=0}=\mathrm{cos}\frac{\pi}{2}x,\quad 
		\left.u_t\right|_{t=0}=\mathrm{cos}\frac{3\pi}{2}x,\quad 0\le x\le l, \\
		&\left.u\right|_{x=0}=0,\quad \left.u\right|_{x=l}=0,\quad t\ge 0.
		\end{aligned}\right.
		\end{array}\] 
	\begin{proof}[Solution]
		设解为$u(x,t)=X(x)T(t)$,则
		$$T^{\prime \prime}X(x)-a^2X^{\prime \prime}(x)T(t)=0,$$
		于是
		$$\frac{X^{\prime \prime}}{X(x)}=\frac{T^{\prime 
				\prime}}{a^2T(t)}=-\lambda. \quad \text{(Constant)}$$
		即
		\begin{align}
		X^{\prime \prime}(x)+\lambda X(x)=0,\label{2-1},\\
		T^{\prime \prime}(x)+\lambda a^2T(t)=0,\label{2-2}
		\end{align}
		由边界条件可知
		$$X(0)T(t)=0,\quad X(l)T(t)=0.$$
		于是
		\begin{align}
		X(0)=X(l)=0,\label{2-3}
		\end{align}
		下面解特征值问题\eqref{2-1}, \eqref{2-2},\\
		情形1.\quad $\lambda<0$,则\eqref{2-1}的通解为:
		\begin{equation*}
		X(x)=c_1e^{\sqrt{-\lambda}x}+c_2e^{-\sqrt{-\lambda}x},
		\end{equation*}
		其中$c_1,c_2$是任意常数,要使它满足边界条件\(X(0)=X(l)=0\),就必须有
		\begin{equation*}
		\left\{\begin{aligned}
		&c_1+c_2=0,\\
		&c_1e^{\sqrt{-\lambda}t}+c_2e^{-\sqrt{-\lambda}t}=0. &\nonumber
		\end{aligned}\right.
		\end{equation*}
		由于其系数行列式\\
		\begin{equation*}
		\begin{vmatrix}
		1 & 1\\
		e^{\sqrt{-\lambda}t} & e^{-\sqrt{-\lambda}t}
		\end{vmatrix}\ne 0
		\end{equation*}
		因此$c_1$和$c_2$必须同时为零,从而$X(x)\equiv 0$.\quad 
		此时特征值问题\eqref{2-1},\eqref{2-3}只有平凡解.\\
		情形2. \quad $\lambda=0$,方程\(X^{\prime \prime}(x)+\lambda X(x)=0\)的通解为
		\begin{equation*}
		X(x)=c_1+c_2x,
		\end{equation*}
		由边界条件\(X(0)=X(l)=0\),得
		\begin{equation*}
		c_1=0,c_1+c_2l=0. 
		\end{equation*}
		所以$c_1=c_2=0$,从而$X(x)\equiv 0$.\quad 此时该特征值问题\eqref{2-1},\eqref{2-3}
		只有
		平凡解.\\
		
		情形3.\quad $\lambda>0$,则\eqref{2-1}的通解为:
		$$X(x)=c_1\mathrm{cos}\sqrt{\lambda}x+c_2\mathrm{sin}\sqrt{\lambda}x.$$
		代入边界条件得
		$$c_1=0,\quad c_2\mathrm{sin}\sqrt{\lambda}l=0,$$
		于是特征值为
		$$\lambda_k=\frac{k^2\pi^2}{l^2},\quad k=0,1,\cdots$$
		对应的特征函数为
		$$X_k(x)=a_k\mathrm{sin}\frac{k\pi}{l} x,\quad k=0,1,\cdots$$
		对特征值$\lambda_k$,解方程\eqref{2-2}得:
		$$T_k(t)=b_k\mathrm{cos}\frac{k\pi}{l} at+c_k 
		\mathrm{sin}\frac{k\pi}{l} at.$$
		于是
		$$u_k(x,t)=\left(A_k\mathrm{cos}\frac{k\pi}{l} at
		+B_k\mathrm{sin}\frac{k\pi}{l} at\right)\mathrm{cos}\frac{k\pi}{l} 
		x,\quad k=0,1,\cdots$$
		作级数
		$$u(x,t)=\sum_{k=0}^{\infty}\left(A_k\mathrm{cos}\frac{k\pi}{l} at
		+B_k\mathrm{sin}\frac{k\pi}{l} at\right)\mathrm{cos}\frac{k\pi}{l} 
		x$$
		代入初始条件,得
		$$\sum_{k=0}^{\infty}A_k\mathrm{cos}\frac{k\pi}{l} 
		x=\mathrm{cos}\frac{\pi}{2}x,\quad \sum_{k=0}^{\infty}B_k\mathrm{cos}\frac{k\pi a}{l} x=\mathrm{cos}\frac{3\pi}{2}x.$$	
		$$\Rightarrow A_1=1,A_0=A_2=\cdots=0,\quad B_0=B_1=B_2=0,B_3=\frac{2}{3\pi 
			a},B_4=B_5=\cdots=0.$$
		所以解为:
		$$u(x,t)=\mathrm{cos}\frac{\pi at}{2}\mathrm{cos}\frac{\pi 
			x}{2}+\frac{2}{3\pi a}\mathrm{sin}\frac{3\pi at}{2}\mathrm{cos}\frac{3\pi 
			x}{2}.
		$$
	\end{proof}
	\end{example}
		
	\begin{example}
		求解下列定解问题
		\begin{equation*}
		\left\{\begin{aligned}
			&u_{xx}+2cosxu_{xy}-sin^{2}xu_{yy}=0, \quad -\infty<x<\infty, 
			y>sinx,  \\
			&\left.u\right|_{y=sinx}=\varphi(x), 
			\left.u_{y}\right|_{y=sinx}=\psi(x), \quad -\infty<x<\infty \\
		\end{aligned}\right.
		\end{equation*}
		\begin{proof}[Solution]
			特征方程为
			$$0=dy^2-2\mathrm{cos}xdxdy-\mathrm{sin}^2xdx^2
			=[dy-(\mathrm{cos}x-1)dx][dy-(\mathrm{cos}x+1)dx]$$
			则
			$$y-\mathrm{sin}x+x=C_1\text{或}y-\mathrm{sin}x-x=C_2.$$
			作自变量变换
			$$\xi=y-\mathrm{sin}x+x\text{或}\eta=y-\mathrm{sin}x-x,$$
			则原方程化为
			$$u_{\xi\eta}=0\Rightarrow 
			u=f(\xi)+g(\eta)=f(y-\mathrm{sin}x+x)+g(y-\mathrm{sin}x-x).$$
			又由初始条件,
			$$f(x)+g(-x)=\varphi(x),$$
			$$f^{\prime}(x)+g^{\prime}(-x)=\phi(x),$$
			$$f(x)-g(-x)=\int_{0}^{x}\phi(t)dt+C,$$
			$$f(x)=\frac{1}{2}[\varphi(x)+\int_{0}^{x}\phi(t)dt+C],$$
			$$g(x)=\frac{1}{2}[\varphi(-x)+\int_{-x}^{0}\phi(t)dt-C],$$
			$$u(x,y)=\frac{1}{2}[\varphi(y-\mathrm{sin}x+x)
			+\varphi(-y+\mathrm{sin}x+x)]+
			\frac{1}{2}\int_{-y+\mathrm{sin}x+x}^{y-\mathrm{sin}x+x}\phi(t)dt.$$
		\end{proof}
	\end{example}
		
	\begin{example}
		求定积分,不详(可能涉及Green函数和Fourier变换)
	\end{example}

	\begin{proof}[Solution]
	部分指数函数积分表
	$$\int e^{cx} dx=\frac{1}{c}e^{cx},\quad \int a^{cx} dx=\frac{1}{cIna}a^{cx} \quad (a>0,a \ne 1)$$
	$$\int xe^{cx}dx=\frac{e^{cx}}{c^2}(cx-1),\quad \int x^2e^{cx}dx=e^{cx}\left(\frac{x^2}{c}-\frac{2x}{c^2}+\frac{2}{c^3}\right)$$
	$$\int x^ne^{cx}dx=\frac{1}{c}x^ne^{cx}-\frac{n}{c}\int x^{n-1}e^{cx}dx,\quad 
	\int xe^{cx^2}dx=\frac{1}{2c}e^{cx^2}$$
	$$\int e^{cx}sinbx dx=\frac{e^{cx}}{c^2+b^2}(c sinbx-bcosbx),\quad \int e^{cx}cosbx dx=\frac{e^{cx}}{c^2+b^2}(c cosbx+bsinbx)$$
	\end{proof}
	
	\begin{example}
		求方程
		\begin{equation}\label{1-4}
		x^2u_{xx}-2xyu_{xy}+y^2u_{yy}+xu_x+yu_y=0
		\end{equation}
		的通解.
	\end{example}
	\begin{proof}[Solution]
		特征方程为
		$$x^2dy^2+2xy dxdy+y^2dx^2=(xdy+ydx)(xdy+ydx)=0,$$
		则
		$$\frac{dy}{dx}=-\frac{y}{x},$$
		进而$$xy=C.\quad (C \text{为常数})$$
		令$\xi=xy,\eta=y$,则
		$$u_x=yu_\xi,\quad u_y=xu_\xi+u_\eta,\quad u_{xy}=u_{\xi} + xyu_{\eta \eta}+yu_{\eta}.$$
		$$u_{xx}=y^2u_{\xi \xi},\quad u_{yy}=x^2u_{\xi \xi}+2xu_{\xi \eta}+u_{\eta \eta},$$
		代入\eqref{1-4}有
		$$y^2u_{\eta \eta}+yu_\eta=0,$$
		情形1. 当$y=0$时,代入方程得:
		$$x^2u_{xx}+xu_x=0,$$
		则$$x=0 \, \text{或} \, xu_{xx}+u_x=0.$$
		令$w=u_x$,则$xw_x+w=0$,解得
		$$\frac{w_x}{w}=-\frac{1}{x} \Rightarrow w=\frac{1}{x}\theta_1(y),$$
		其中$\theta_1(y)$是任意二阶连续可导函数.
		即$$u_x=\frac{1}{x}\theta_1(y) \Rightarrow u(x,y)=\theta_1(y) {\rm In} x+\theta_2(y),$$
		其中$\theta_2(y)$也是任意二阶连续可导函数.
		
		情形2. 当$y\ne 0$时,代入方程得:
		$$yu_{\eta \eta}+u_{\eta}=0,$$
		令$w=u_\eta$,则$\eta w_\eta +w=0$,解得
		$$\frac{w_x}{w}=-\frac{1}{\eta} \Rightarrow w=\frac{1}{\eta}\theta_1(\xi),$$
		其中$\theta_1(\xi)$是任意二阶连续可导函数.
		即$$u_\eta=\frac{1}{\eta}\theta_1(\xi) \Rightarrow u(\xi,\eta)=\theta_1(\xi) {\rm In} \eta+\theta_2(\xi),$$
		其中$\theta_2(\xi)$也是任意二阶连续可导函数,则
		$$u(x,y)=\theta_1(xy){\rm In}y+\theta_2(xy).$$
		\end{proof}
		
		

\end{document}
